\documentclass{article}
\usepackage{xcolor}
\usepackage{textcomp}
\usepackage{hyperref}

\title{Research of Parameters in Strategies of the Iterated Continuous Prisoner's Dilemma}
\author{Adrian Hossner}
\date{ } % display no date

\begin{document}
\maketitle

\newpage
\tableofcontents
\newpage

\section*{Abstract}

\section{Introduction}

\begin{itemize}

	% Key idea:
	% Much uncertainty spread out because of (far-)right populists.
	% With insecurity comes random behaviour.
	% This simulations shows the difference of a random and a determined strategy.

	\item Interactions\\
		implication range is wide:\\
		from two persons in a relationship to countries in alliance\\
		from two impala's algrooming to zebras warning the herd about a lion \\
	% https://www.cell.com/trends/ecology-evolution/pdf/S0169-5347(00)88988-0.pdf

	\item Key hook:\\
		helping someone with possible cost -- cooperation\\
		being selfish but breaking trust -- defection\\

	\item model PD:\\
		single interaction\\
		already well analysed\\
		Nash found solution/equilibrium\\

	\item Iterated:\\
		more interesting, insightful\\
		strategies\\

	\item non-suited variants:\\
		Axelrod's Tournament\\
		Evolution\\

	\item continuous:\\
		more complex, accurate\\

	\item parameters:\\
		determines behaviour\\
		create surface\\
		the part that has not been experimented\\

	\item overall and difference:\\
		overall: wealth of population\\
		difference: competence between individuals\\

\end{itemize}

\section{Theoretical Foundations}
\begin{itemize}

	\item Game Theory (optional, not needed for understanding)\\
		mathematical framework to investigate interactions\\

	\item Prisoner's Dilemma:\\
		
The Prisoner's Dilemma (PD) is a famous intellectual game in Game Theory. 
It was invented by Merrill Flood and Melvin Dresher in 1950. 
In order to understand the dilemma, a proper context has to be established. 
Two persons are arrested and are sued for having stolen something. 
The police, however, has not enough evidence to imprison them. 
As a consequence of that, the police interrogates the two criminals. 
They can either say nothing or confess that the other suspect was involved in the crime. 
They can either cooperate or defect without having the possibility to talk to each other before the decision. 
A certain tactic is being used by the police officers. 
They make it profitable for the criminals to confess. 
They let the criminal out of prison earlier if he/she confesses. 
But if the other also confesses, both stay long in prison. 
If he/she says nothing and the other confesses, he/she will stay very long in prison. 
If, however, both stay silent, both of them go to prison in a relatively short amount of time. 
The following matrix visualises the pay-off's in this game in a comprehensive way:

% specific pay-off matrix
\begin{center}
\begin{tabular}{ c|c|c }
   & C & D \\ 
   \hline
 C & 3, 3 & 10, 0\\  
   \hline
 D & 0, 10 & 5, 5
\end{tabular}
\end{center}

C stands for cooperation and D stands for defection. 
Cooperation is defined by remaining silent whereas defection means confessing. 
The dilemma consists of the following. 
Confessing seems attractive since the interrogated criminal can walk out freely without going to prison. 
However, if the second criminal also confesses, both get five years in prison. 
This is, nevertheless, the worst outcome for both which could have been avoided as both could have stayed silent.\\
John Nash, a mathematician which made great contributions in Game Theory, has proved that it is the most logical option for both to confess, always. 
A Nash-equilibrium is introduced. 
It defines a stable state in which both would not change their decision even though they would know if the other cooperated or defected.\\
This game composes a one-time interaction very well in a mathematical and analytical perspective. 
The pay-off's in the matrix can be changed as long as these rules stay valid.
% \url{https://www.investopedia.com/terms/n/nash-equilibrium.asp}


% general pay-off matrix
\begin{center}
\begin{tabular}{ c|c|c }
   & C & D \\ 
   \hline
 C & R, R & S, T\\  
   \hline
 D & T, S & P, P
\end{tabular}
\end{center}

$$T > R > P > S$$
$$2*R > T + S$$

	\item Iterated:\\
		
The Iterated Prisoner's Dilemma (IPD) is a game which extends the PD. 
As the name suggests, the PD is played a number of times sequentially. 
After each round the pay-off gets accumulated to the points one strategy has already gained. 
Since a strategy wants to gain as many points as possible, the pay-off matrix should be altered accordingly.\\

\begin{center}
\begin{tabular}{ c|c|c }
   & C & D \\ 
   \hline
 C & 3, 3 & 0, 5\\  
   \hline
 D & 5, 0 & 1, 1
\end{tabular}
\end{center}

The Nash-equilibrium in the IPD differs from that in the PD.
This means that cooperation can earn out long-term. 
By cooperating, one is building trust with another. 
One strategy can only get the maximum points, which can be gained by always getting the 5 points each round, by defecting and the opponent's cooperation. 
Since the opponent will likely not cooperate after one strategy always defects, the maximum is practically impossible to receive. 
So, the relative high amount and the possibility of getting it for both is by cooperating.\\
The number of round that will be played has to be unknown to both strategies. 
There is no sense in cooperating the last round since trust does not matter after the game is ended. 
And because trust will be given up in the last round, there is no sense for both to keep trust the second last round. 
This will end up in an inductive defection behaviour for both strategies the whole game.\\
This extension of the PD allows us to simulate a relationship over time where the two players interact multiple times.

	\item Strategies:\\

A strategy is to be considered as an autonomous player which has its own behaviour.
A strategy has the whole history of its own contribution and the ones of the opponent at hand to generate the decision for the next round.
These decisions are calculated by using conditions and probabilities on the given data.\\

	\item Continuous:\\

The continuous variant is considered more accurate to the real world since the decisions are rather rare only either cooperation or defection.
The pay-off system needs to be altered accordingly.
The pay-off of [sciencedirect.com] is suitable for this project.

$$p1 = y - c*x$$
$$p2 = x - c*y$$

The amount of cooperation is called investment and can range from zero to one (0 to 1) where full defection is represented by 0 and cooperation corresponds to 1.
The variable x is the investment of strategy 1 and y is the investment of strategy 2.
The parameter c also ranges from 0 to 1 and describes the cost of cooperation.
\\

		% 0 to 1 rather than cooperation or defection\\
		% Pay-off system\\ % https://www.sciencedirect.com/science/article/pii/S0022519306004255?casa_token=hIA0lYvzjf8AAAAA:Up2uQ89wotaLz7s1R0dM2FK7SaulAo40wIM-BdM9yooZ8uJeRL6mPs-K55dPNGs4XkcclNVjDQ#aep-section-id24
		% more accurate\\
		% more complex, generates more data \textrightarrow more insightful\\

	\item Noise:\\

Noise in the ICPD is the equivalent to miscommunication or misunderstanding in the real world.
In this variant of the PD, noise is essential to trigger interesting outcomes.
When a game of the ICPD is started and two strategies who start with a full investment, many strategies only respond with full investments.
So noise is necessary to trigger continuous investments. 
Without noise, in many cases, the ICPD would be equivalent to the IPD.

		% sometimes necessary to trigger continuous investments\\
		% simulates misunderstandings\\

	\item Parameter-based:\\

The main idea of this paper is to use parameter-based strategies.
These strategies hold a parameter that defines the behaviour of the strategy.
The parameter will always range from 1 to 10 inclusively.
[Example]


		% Define parameter within strategies\\
		% determine the behaviour\\

	\item Surfaces:\\

The end result of this paper will be several surface plots.
The data will be generated by letting two parameter-based strategies play against each other the ICPD.
The game will be structured so that every parameter came against every other parameter in the ICPD.
Like this a surface can be plotted by having the x-axis being the parameter of strategy 1 and the y-axis being the parameter of strategy 2.
The z-axis will indicate the points one strategy gained.
That being said, one surface will be shown for each strategy.
Further more, the two z-axis of the two surfaces can be added and form a new surface which shows the overall points gained by both strategies.
The complement to that would be to subtract the two z-axis to plot a surface which describes how much better one strategy was than the other.

		% Let two strategies play ICPD\\
		% let every parameter play against every other parameter\\
		% parameters as x and y-axis, points as z-axis\\
		% one surface for each strategy\\
		% overall surface is population wealth (addition)\\
		% difference surface is individual competence (subtraction)

	\item Simultaneous vs Alternating:\\

Two main differences can be seen when simulating an iterated variant of the PD.
On one hand there is the simultaneous form whereby the strategies submit their contribution at the same time.
On the other hand, there is the alternating form in which on e strategy submits its contribution and the other strategy can respond to this contribution is that round.
Since the alternating form gives and advantage to the strategy which is allowed to respond, this only makes things more complicated than necessary.
So, this analysis is only dedicated to the simultaneous form.

	\item Current Findings:\\

In the simpler variants, e.g. the IPD without noise, it has been proved many times over that the strategy tit-for-tat is the most successful one. [sources]
The success of strategies, however, can vary depending on certain conditions such as the influence of noise or different pay-off systems where continuous investments can be submitted.
So, the very specific area of the simultaneous ICPD, has not been very well explored.

		% (None really)\\
		% Not well explored: simultaneous Iterated Continuous Prisoner's Dilemma

\end{itemize}

\section{Methods and Implementation}
\begin{itemize}

	\item Simulated Strategies and explanation of parameter:\\
		Average:\\
The strategy Average belongs to the group of the responding/responsive strategies.
It takes the investments of the opponent in the last $n$ previous rounds where $n = parameter$ as variables to its function to calculate the mean.
This mean will then be submitted in the next round.
Since in the first $n$ rounds the strategy cannot calculate an average, it will submit full cooperation to offer maximum points for both of them.



		RandomContinuous\\
		RandomDiscrete\\
		Adapt\\
		...\\

	\item Programming language:\\

The simulation, data generation and visualisation is completely written in Python (3).

% Everything in Python (simulation/data generation/visualisation)\\
		% Github: \href{https://github.com/adho08/Prisoner-s-Dilemma}{Repository}\\

	\item Data Generation:\\
		two strategies\\
		every parameter against every parameter\\
		play several times to smooth out noise (average)\\

	\item Implemented Variants:\\
		Prisoner's Dilemma\\
		Iterated Prisoner's Dilemma\\
		Iterated Continuous Prisoner's Dilemma\\
		Parameter-Based Strategies\\
		only PBS important for MA\\
	
\end{itemize}

\section{Results}
\begin{itemize}

	\item Surfaces:\\
		plots\\
		surface of strategy 1\\
		surface of strategy 2\\
		surface of both (added)\\
		surface showing difference (subtracted)\\

	\item Analysis:\\
		hills and valleys\\

\end{itemize}

\section{Discussion}
\begin{itemize}

	\item ?

\end{itemize}

\section{Conclusions and Outlook}
\begin{itemize}

	\item Summary and Quintessence\\
		which strategy 1 better than which strategy 2\\
		which parameter to be used\\

	\item Application in the Real World\\
		strategies translated to real persons\\
		giving examples (Politics, economics to simple human beings)\\
		prospect to the future (which ones will win?)

\end{itemize}

\section{Self Reflection}
\begin{itemize}

	\item Challenges\\
		estimation of time consumption\\
		transforming ideas during process\\
		dedication

	\item Learnings\\
		reading professional papers\\
		writing English\\
		experimental workflow (usage of neovim, tmux and terminal)

\end{itemize}

\section{Annex}
\end{document}
